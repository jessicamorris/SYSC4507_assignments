\documentclass{article}

\usepackage[scientific-notation=true, binary-units=true]{siunitx}
\sisetup{per-mode=fraction}%
\sisetup{scientific-notation=false}%

\title{SYSC 4507 Assignment 1}
\date{February 7th, 2017}
\author{Jessica Morris \(100882290\)}

\begin{document}

\maketitle

\begin{enumerate}

\item
\begin{enumerate}
\item
$$ \text{\# of instructions} = \frac{ \text{execution time} }{ \text{CPI} \times \text{seconds/cycle} } $$
$$ \text{\# of instructions} = \frac{ \SI{1.8}{\second} }{ 1.6 \times \frac{1}{ \SI{900}{\mega\hertz} }} $$
$$ \text{\# of instructions} = 1012500000 $$

\item
$$ \text{MIPS} = \frac{1012500000 \text{ instructions}}{\SI{1.8}{\second}} $$
$$ \text{MIPS} = 562500000 $$

\item Using a processor that has a faster MIPS rating should result in a positive percent speedup. Since a higher MIPS-rated processor will execute the program faster, $ T_{new} < T_{old} $, resulting in $ \frac{T_{old} - T_{new}}{T_{new}} $ being positive.

\end{enumerate}

\item Loop 1 executes $ n $ times. Loop 2 executes $ n $ times, loop 3 will execute $ n \times (n - j)  $ times, and loop 4 will execute $ n \times (n - j) \times n $ times. The expression for the algorithm's run-time is then:

$$ n + n^3 -jn^2 $$

The highest-order variable in this expression is $ n^3 $, so this code fragment is of order $ O(n^3) $.

\end{enumerate}
\end{document}
